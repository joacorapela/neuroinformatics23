\documentclass{article}

\usepackage[hypertexnames=false,colorlinks=true,breaklinks]{hyperref}
\usepackage{graphicx}
\usepackage[shortlabels]{enumitem}
\usepackage{subcaption}
\usepackage{caption}

\title{Report worksheet 1}
\author{Joaqu\'{i}n Rapela}

\begin{document}

\maketitle

\section*{Exercise 1: t-test for non-Gaussian distributions}

\begin{enumerate}[(a)]

    \item  Please refer to Figure~\ref{fig:ex1a}.

        \begin{figure}
            \begin{center}
                \href{https://www.gatsby.ucl.ac.uk/~rapela/neuroinformatics/2023/ws1/figures/ex1_distributionNormal_popmean0.0000_mean0.0000_nSamples10000.html}{\includegraphics[width=4in]{../figures/ex1_distributionStdCauchy_mean0.0000_nSamples10000.png}}

                \caption{Exercise 1a. Histogram of p-values of 1.000 t-tests
                evaluating if the mean of 10.000 samples from a $\mathcal{N}(0,
                1)$ is equal to zero.
                Click on the figure to see its interactive version.
                The code to generate this figure appears
                \href{https://github.com/joacorapela/neuroinformatics23/blob/master/worksheets/ws1/mySolution/code/scripts/doEx1.py}{here} and the
                parameters used for this script appear
                \href{https://github.com/joacorapela/neuroinformatics23/blob/master/worksheets/ws1/mySolution/code/scripts/doEx1a.csh}{here}.}

                \label{fig:ex1a}

            \end{center}
        \end{figure}

    \item  Please refer to Figure~\ref{fig:ex1b}.

        \begin{figure}
            \begin{center}

                \begin{subfigure}{1.0\textwidth}
                    \centering
                    \href{https://www.gatsby.ucl.ac.uk/~rapela/neuroinformatics/2023/ws1/figures/ex1_distributionNormal_mean0.1000_nSamples1000.html}{\includegraphics[width=4in]{../figures/ex1_distributionNormal_mean0.1000_nSamples1000.png}}
                    \caption{Histogram of p-values of 1.000 t-tests evaluating if the mean of 10.000 samples from a $\mathcal{N}(0.1, 1)$ is equal to zero.  Click on the figure to see its interactive version.}
                    \label{fig:ex1b_1}
                \end{subfigure}

                \begin{subfigure}{1.0\textwidth}
                    \centering
                    \href{https://www.gatsby.ucl.ac.uk/~rapela/neuroinformatics/2023/ws1/figures/ex1_distributionNormal_mean0.0100_nSamples1000.html}{\includegraphics[width=4in]{../figures/ex1_distributionNormal_mean0.0100_nSamples1000.png}}
                    \caption{Histogram of p-values of 1.000 t-tests evaluating if the mean
                    of 10.000 samples from a $\mathcal{N}(0.01, 1)$ is equal to zero.
                    Click on the figure to see its interactive version.}
                    \label{fig:ex1b_2}
                \end{subfigure}

                \caption{Exercise 1b.
                The code to generate this figure appears
                \href{https://github.com/joacorapela/neuroinformatics23/blob/master/worksheets/ws1/mySolution/code/scripts/doEx1.py}{here} and the
                parameters used for this script appear
                \href{https://github.com/joacorapela/neuroinformatics23/blob/master/worksheets/ws1/mySolution/code/scripts/doEx1b.csh}{here}.}
                \label{fig:ex1b}

            \end{center}
        \end{figure}

    \item  Please refer to Figure~\ref{fig:ex1c}.

        \begin{figure}
            \begin{center}

                \begin{subfigure}{1.0\textwidth}
                    \centering
                    \href{https://www.gatsby.ucl.ac.uk/~rapela/neuroinformatics/2023/ws1/figures/ex1_distributionStdCauchy_mean0.0000_nSamples10000.html}{\includegraphics[width=4in]{../figures/ex1_distributionStdCauchy_mean0.0000_nSamples10000.png}}

                    \caption{Histogram of p-values of 1.000 t-tests evaluating
                    if the mean of 10.000 samples from a standard Cauchy
                    distribution is equal to zero.  Click on the figure to see
                    its interactive version.}

                    \label{fig:ex1c_1}
                \end{subfigure}

                \begin{subfigure}{1.0\textwidth}
                    \centering
                    \href{https://www.gatsby.ucl.ac.uk/~rapela/neuroinformatics/2023/ws1/figures/ex1_distributionStdCauchy_mean0.0000_nSamples3.html}{\includegraphics[width=4in]{../figures/ex1_distributionStdCauchy_mean0.0000_nSamples3.png}}

                    \caption{Histogram of p-values of 1.000 t-tests evaluating if the mean
                    of 3 samples from a standard Cauchy distribution is equal to zero.
                    Click on the figure to see its interactive version.}

                    \label{fig:ex1c_2}
                \end{subfigure}

                \caption{Exercise 1c.
                The code to generate this figure appears
                \href{https://github.com/joacorapela/neuroinformatics23/blob/master/worksheets/ws1/mySolution/code/scripts/doEx1.py}{here} and the
                parameters used for this script appear
                \href{https://github.com/joacorapela/neuroinformatics23/blob/master/worksheets/ws1/mySolution/code/scripts/doEx1c.csh}{here}.}
                \label{fig:ex1c}

            \end{center}
        \end{figure}

    \item  Please refer to Figure~\ref{fig:ex1d}.

        \begin{figure}
            \begin{center}

                \begin{subfigure}{1.0\textwidth}
                    \centering
                    \href{https://www.gatsby.ucl.ac.uk/~rapela/neuroinformatics/2023/ws1/figures/ex1_distributionRademacher_mean0.0000_nSamples10000.html}{\includegraphics[width=4in]{../figures/ex1_distributionRademacher_mean0.0000_nSamples10000.png}}

                    \caption{Histogram of p-values of 1.000 t-tests evaluating
                    if the mean of 10.000 samples from a Rademacher
                    distribution is equal to zero.  Click on the figure to see
                    its interactive version.}

                    \label{fig:ex1d_1}
                \end{subfigure}

                \begin{subfigure}{1.0\textwidth}
                    \centering
                    \href{https://www.gatsby.ucl.ac.uk/~rapela/neuroinformatics/2023/ws1/figures/ex1_distributionRademacher_mean0.0000_nSamples3.html}{\includegraphics[width=4in]{../figures/ex1_distributionRademacher_mean0.0000_nSamples3.png}}

                    \caption{Histogram of p-values of 1.000 t-tests evaluating if the mean
                    of 3 samples from a Rademacher distribution is equal to zero.
                    Click on the figure to see its interactive version.}

                    \label{fig:ex1d_2}
                \end{subfigure}

                \caption{Exercise 1d.
                The code to generate this figure appears
                \href{https://github.com/joacorapela/neuroinformatics23/blob/master/worksheets/ws1/mySolution/code/scripts/doEx1.py}{here} and the
                parameters used for this script appear
                \href{https://github.com/joacorapela/neuroinformatics23/blob/master/worksheets/ws1/mySolution/code/scripts/doEx1d.csh}{here}.}
                \label{fig:ex1d}

            \end{center}
        \end{figure}

    \item  Please refer to Figure~\ref{fig:ex1e}.

        \begin{figure}
            \begin{center}

                \begin{subfigure}{1.0\textwidth}
                    \centering
                    \href{https://www.gatsby.ucl.ac.uk/~rapela/neuroinformatics/2023/ws1/figures/ex1_distributionVerySkewed_popmean0.0010_mean0.0000_nSamples100.html}{\includegraphics[width=4in]{../figures/ex1_distributionVerySkewed_popmean0.0010_mean0.0000_nSamples100.png}}

                    \caption{Histogram of p-values of 1.000 t-tests evaluating
                    if the mean of 100 samples from the very skewed
                    distribution distribution is equal to 0.001.  Click on the
                    figure to see its interactive version.}

                    \label{fig:ex1e_1}
                \end{subfigure}

                \begin{subfigure}{1.0\textwidth}
                    \centering
                    \href{https://www.gatsby.ucl.ac.uk/~rapela/neuroinformatics/2023/ws1/figures/ex1_distributionRademacher_mean0.0000_nSamples3.html}{\includegraphics[width=4in]{../figures/ex1_distributionRademacher_mean0.0000_nSamples3.png}}

                    \caption{Histogram of p-values of 1.000 t-tests evaluating if the mean
                    of 3 samples from the very skewed distribution is equal to zero.
                    Click on the figure to see its interactive version.}

                    \label{fig:ex1e_2}
                \end{subfigure}

                \caption{Exercise 1e.
                The code to generate this figure appears
                \href{https://github.com/joacorapela/neuroinformatics23/blob/master/worksheets/ws1/mySolution/code/scripts/doEx1.py}{here} and the
                parameters used for this script appear
                \href{https://github.com/joacorapela/neuroinformatics23/blob/master/worksheets/ws1/mySolution/code/scripts/doEx1e.csh}{here}.}
                \label{fig:ex1e}

            \end{center}
        \end{figure}

\end{enumerate}

\section*{Exercise 2: randomization test}

Please refer to Figure~\ref{fig:ex2}.

\begin{figure}
    \begin{center}

        \begin{subfigure}{1.0\textwidth}
            \centering
            \href{https://www.gatsby.ucl.ac.uk/~rapela/neuroinformatics/2023/ws1/figures/ex2_distributionRademacher_popmean0.0000_mean0.0000_nSamples10.html}{\includegraphics[width=4in]{../figures/ex2_distributionRademacher_popmean0.0000_mean0.0000_nSamples10.png}}

            \caption{Histogram of p-values of 1.000 randomization tests evaluating
            if the mean of 10 samples from the Rademacher
            distribution distribution is equal to 0.0.  Click on the
            figure to see its interactive version.}

            \label{fig:ex2_1}
        \end{subfigure}

        \begin{subfigure}{1.0\textwidth}
            \centering
            \href{https://www.gatsby.ucl.ac.uk/~rapela/neuroinformatics/2023/ws1/figures/ex2_distributionSkewed_popmean0.0000_mean0.0000_nSamples10.html}{\includegraphics[width=4in]{../figures/ex2_distributionSkewed_popmean0.0000_mean0.0000_nSamples10.png}}

            \caption{Histogram of p-values of 1.000 randomization tests evaluating if the mean
            of 10 samples from the skewed distribution is equal to zero.
            Click on the figure to see its interactive version.}

            \label{fig:ex2_2}
        \end{subfigure}

        \caption{Exercise 2. The code to generate this figure appears
        \href{https://github.com/joacorapela/neuroinformatics23/blob/master/worksheets/ws1/mySolution/code/scripts/doEx2.py}{here}
        and the parameters used for this script appear
        \href{https://github.com/joacorapela/neuroinformatics23/blob/master/worksheets/ws1/mySolution/code/scripts/doEx2.csh}{here}.
        }

        \label{fig:ex2}

    \end{center}
\end{figure}

\section*{Exercise 3: raster plots}

Please refer to Figure~\ref{fig:ex3}.

\begin{figure}
    \begin{center}

        \begin{subfigure}{1.0\textwidth}
            \centering
            \href{https://www.gatsby.ucl.ac.uk/~rapela/neuroinformatics/2023/ws1/figures/spikes_times_clusterID41_epochedBystimOn_times_sortedByresponse_times_colorschoice.html}{\includegraphics[width=4in]{../figures/spikes_times_clusterID41_epochedBystimOn_times_sortedByresponse_times_colorschoice.png}}

            \caption{Rasterplot of neuron 41 aligned to
            \texttt{stimOn\_times},
            sorted by
            \texttt{response\_times}
            and colored by
            % \texttt{choice}.
            }

            \label{fig:ex3_1}
        \end{subfigure}

        \begin{subfigure}{1.0\textwidth}
            \centering
            \href{https://www.gatsby.ucl.ac.uk/~rapela/neuroinformatics/2023/ws1/figures/spikes_times_clusterID41_epochedBystimOn_times_sortedByresponse_times_colorsfeedbackType.html}{\includegraphics[width=4in]{../figures/spikes_times_clusterID41_epochedBystimOn_times_sortedByresponse_times_colorsfeedbackType.png}}

            \caption{Rasterplot of neuron 41 aligned to
            \texttt{stimOn\_times},
            sorted by
            \texttt{response\_times}
            and colored by
            \texttt{feedbackType}.
            }

            \label{fig:ex3_2}
        \end{subfigure}

        \caption{Exercise 3. The code to generate this figure appears
        \href{https://github.com/joacorapela/neuroinformatics23/blob/master/worksheets/ws1/mySolution/code/scripts/doPlotEpochedSpikesTimes.py}{here}
        and the parameters used for this script appear
        \href{https://github.com/joacorapela/neuroinformatics23/blob/master/worksheets/ws1/mySolution/code/scripts/doPlotEpochedSpikesTimes.csh}{here}.
        }

        \label{fig:ex3}


    \end{center}
\end{figure}


\end{document}
