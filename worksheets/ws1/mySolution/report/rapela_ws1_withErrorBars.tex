\documentclass{article}

\usepackage[hypertexnames=false,colorlinks=true,breaklinks]{hyperref}
\usepackage{graphicx}
\usepackage[shortlabels]{enumitem}
\usepackage{subcaption}
\usepackage{caption}
\usepackage{float}
\usepackage{amsmath}
\usepackage{amsthm}
\usepackage[title]{appendix}

\newtheorem{claim}{Claim}
\newtheorem{lemma}{Lemma}

\title{Report worksheet 1}
\author{Joaqu\'{i}n Rapela}

\begin{document}

\maketitle

\section*{Exercise 1: t-test for non-Gaussian distributions}

Under the null hypothesis, p-values should follow a uniform distribution (see
proof in Apendix~\ref{sec:pValuesAreUniform}). Therefore, when sampling from a
normal distribution with zero mean, we should observe
0.05*\texttt{n\_repeats}=50 tests with pvalues in the range $[p,
p+0.05],\;\forall p\in[0, 0.95]$. In particular, when sampling from a Normal
distribution with zero mean, we should observe 50 tests with p\_value\textless
0.05.

\begin{enumerate}[(a)]

    \item  Please refer to Figure~\ref{fig:ex1a}. Here we are sampling from a
        normal distribution with zero mean. Thus, all histogram bins of length
        0.05 should have around 50 counts. And this is what
        Figure~\ref{fig:ex1a} shows.

        \begin{figure}[H]
            \begin{center}
                \href{https://www.gatsby.ucl.ac.uk/~rapela/neuroinformatics/2023/ws1/figures/ex1_distributionNormal_popmean0.0000_mean0.0000_nSamples10000_withErrorBars.html}{\includegraphics[width=4in]{../figures/ex1_distributionNormal_popmean0.0000_mean0.0000_nSamples10000_withErrorBars.png}}

                \caption{Exercise 1a. Histogram of p-values of 1.000 t-tests
                evaluating if the mean of 10.000 samples from a $\mathcal{N}(0,
                1)$ is equal to zero.
                Click on the figure to see its interactive version.
                The code to generate this figure appears
                \href{https://github.com/joacorapela/neuroinformatics23/blob/master/worksheets/ws1/mySolution/code/scripts/doEx1WithErrorBars.py}{here} and the
                parameters used for this script appear
                \href{https://github.com/joacorapela/neuroinformatics23/blob/master/worksheets/ws1/mySolution/code/scripts/doEx1a.csh}{here}.}

                \label{fig:ex1a}

            \end{center}
        \end{figure}

    \item  Please refer to Figure~\ref{fig:ex1b}.

        \begin{figure}[H]
            \begin{center}

                \begin{subfigure}{1.0\textwidth}
                    \centering
                    \href{https://www.gatsby.ucl.ac.uk/~rapela/neuroinformatics/2023/ws1/figures/ex1_distributionNormal_popmean0.0000_mean0.1000_nSamples10000_withErrorBars.html}{\includegraphics[width=4in]{../figures/ex1_distributionNormal_popmean0.0000_mean0.1000_nSamples10000_withErrorBars.png}}
                    \caption{Histogram of p-values of 1.000 t-tests evaluating if the mean of 10.000 samples from a $\mathcal{N}(0.1, 1)$ is equal to zero.  Click on the figure to see its interactive version.}
                    \label{fig:ex1b_1}
                \end{subfigure}

                \begin{subfigure}{1.0\textwidth}
                    \centering
                    \href{https://www.gatsby.ucl.ac.uk/~rapela/neuroinformatics/2023/ws1/figures/ex1_distributionNormal_popmean0.0000_mean0.0100_nSamples10000_withErrorBars.html}{\includegraphics[width=4in]{../figures/ex1_distributionNormal_popmean0.0000_mean0.0100_nSamples10000_withErrorBars.png}}
                    \caption{Histogram of p-values of 1.000 t-tests evaluating if the mean
                    of 10.000 samples from a $\mathcal{N}(0.01, 1)$ is equal to zero.
                    Click on the figure to see its interactive version.}
                    \label{fig:ex1b_2}
                \end{subfigure}

                \caption{Exercise 1b.
                The code to generate this figure appears
                \href{https://github.com/joacorapela/neuroinformatics23/blob/master/worksheets/ws1/mySolution/code/scripts/doEx1WithErrorBars.py}{here} and the
                parameters used for this script appear
                \href{https://github.com/joacorapela/neuroinformatics23/blob/master/worksheets/ws1/mySolution/code/scripts/doEx1bWithErrorBars.csh}{here}.}
                \label{fig:ex1b}

            \end{center}
        \end{figure}

    \item  Please refer to Figure~\ref{fig:ex1c}.

        \begin{figure}[H]
            \begin{center}

                \begin{subfigure}{1.0\textwidth}
                    \centering
                    \href{https://www.gatsby.ucl.ac.uk/~rapela/neuroinformatics/2023/ws1/figures/ex1_distributionStdCauchy_popmean0.0000_mean0.0000_nSamples10000_withErrorBars.html}{\includegraphics[width=4in]{../figures/ex1_distributionStdCauchy_popmean0.0000_mean0.0000_nSamples10000_withErrorBars.png}}

                    \caption{Histogram of p-values of 1.000 t-tests evaluating
                    if the mean of 10.000 samples from a standard Cauchy
                    distribution is equal to zero.  Click on the figure to see
                    its interactive version.}

                    \label{fig:ex1c_1}
                \end{subfigure}

                \begin{subfigure}{1.0\textwidth}
                    \centering
                    \href{https://www.gatsby.ucl.ac.uk/~rapela/neuroinformatics/2023/ws1/figures/ex1_distributionStdCauchy_popmean0.0000_mean0.0000_nSamples3_withErrorBars.html}{\includegraphics[width=4in]{../figures/ex1_distributionStdCauchy_popmean0.0000_mean0.0000_nSamples3_withErrorBars.png}}

                    \caption{Histogram of p-values of 1.000 t-tests evaluating if the mean
                    of 3 samples from a standard Cauchy distribution is equal to zero.
                    Click on the figure to see its interactive version.}

                    \label{fig:ex1c_2}
                \end{subfigure}

                \caption{Exercise 1c.
                The code to generate this figure appears
                \href{https://github.com/joacorapela/neuroinformatics23/blob/master/worksheets/ws1/mySolution/code/scripts/doEx1WithErrorBars.py}{here} and the
                parameters used for this script appear
                \href{https://github.com/joacorapela/neuroinformatics23/blob/master/worksheets/ws1/mySolution/code/scripts/doEx1cWithErrorBars.csh}{here}.}
                \label{fig:ex1c}

            \end{center}
        \end{figure}

    \item  Please refer to Figure~\ref{fig:ex1d}.

        \begin{figure}[H]
            \begin{center}

                \begin{subfigure}{1.0\textwidth}
                    \centering
                    \href{https://www.gatsby.ucl.ac.uk/~rapela/neuroinformatics/2023/ws1/figures/ex1_distributionRademacher_popmean0.0000_mean0.0000_nSamples10000_withErrorBars.html}{\includegraphics[width=4in]{../figures/ex1_distributionRademacher_popmean0.0000_mean0.0000_nSamples10000_withErrorBars.png}}

                    \caption{Histogram of p-values of 1.000 t-tests evaluating
                    if the mean of 10.000 samples from a Rademacher
                    distribution is equal to zero.  Click on the figure to see
                    its interactive version.}

                    \label{fig:ex1d_1}
                \end{subfigure}

                \begin{subfigure}{1.0\textwidth}
                    \centering
                    \href{https://www.gatsby.ucl.ac.uk/~rapela/neuroinformatics/2023/ws1/figures/ex1_distributionRademacher_popmean0.0000_mean0.0000_nSamples3_withErrorBars.html}{\includegraphics[width=4in]{../figures/ex1_distributionRademacher_popmean0.0000_mean0.0000_nSamples3_withErrorBars.png}}

                    \caption{Histogram of p-values of 1.000 t-tests evaluating if the mean
                    of 3 samples from a Rademacher distribution is equal to zero.
                    Click on the figure to see its interactive version.}

                    \label{fig:ex1d_2}
                \end{subfigure}

                \caption{Exercise 1d.
                The code to generate this figure appears
                \href{https://github.com/joacorapela/neuroinformatics23/blob/master/worksheets/ws1/mySolution/code/scripts/doEx1WithErrorBars.py}{here} and the
                parameters used for this script appear
                \href{https://github.com/joacorapela/neuroinformatics23/blob/master/worksheets/ws1/mySolution/code/scripts/doEx1d.csh}{here}.}
                \label{fig:ex1d}

            \end{center}
        \end{figure}

    \item  Please refer to Figure~\ref{fig:ex1e}.

        \begin{figure}[H]
            \begin{center}

                \begin{subfigure}{1.0\textwidth}
                    \centering
                    \href{https://www.gatsby.ucl.ac.uk/~rapela/neuroinformatics/2023/ws1/figures/ex1_distributionVerySkewed_popmean0.0010_mean0.0000_nSamples100_withErrorBars.html}{\includegraphics[width=4in]{../figures/ex1_distributionVerySkewed_popmean0.0010_mean0.0000_nSamples100_withErrorBars.png}}

                    \caption{Histogram of p-values of 1.000 t-tests evaluating
                    if the mean of 100 samples from the very skewed
                    distribution distribution is equal to 0.001.  Click on the
                    figure to see its interactive version.}

                    \label{fig:ex1e_1}
                \end{subfigure}

                \begin{subfigure}{1.0\textwidth}
                    \centering
                    \href{https://www.gatsby.ucl.ac.uk/~rapela/neuroinformatics/2023/ws1/figures/ex1_distributionRademacher_popmean0.0000_mean0.0000_nSamples3_withErrorBars.html}{\includegraphics[width=4in]{../figures/ex1_distributionRademacher_popmean0.0000_mean0.0000_nSamples3_withErrorBars.png}}

                    \caption{Histogram of p-values of 1.000 t-tests evaluating if the mean
                    of 3 samples from the very skewed distribution is equal to zero.
                    Click on the figure to see its interactive version.}

                    \label{fig:ex1e_2}
                \end{subfigure}

                \caption{Exercise 1e.
                The code to generate this figure appears
                \href{https://github.com/joacorapela/neuroinformatics23/blob/master/worksheets/ws1/mySolution/code/scripts/doEx1WithErrorBars.py}{here} and the
                parameters used for this script appear
                \href{https://github.com/joacorapela/neuroinformatics23/blob/master/worksheets/ws1/mySolution/code/scripts/doEx1e.csh}{here}.}
                \label{fig:ex1e}

            \end{center}
        \end{figure}

\end{enumerate}

\section*{Exercise 2: randomization test}

Please refer to Figure~\ref{fig:ex2}.

\begin{figure}[H]
    \begin{center}

        \begin{subfigure}{1.0\textwidth}
            \centering
            \href{https://www.gatsby.ucl.ac.uk/~rapela/neuroinformatics/2023/ws1/figures/ex2_distributionRademacher_popmean0.0000_mean0.0000_nSamples10.html}{\includegraphics[width=4in]{../figures/ex2_distributionRademacher_popmean0.0000_mean0.0000_nSamples10.png}}

            \caption{Histogram of p-values of 1.000 randomization tests evaluating
            if the mean of 10 samples from the Rademacher
            distribution distribution is equal to 0.0.  Click on the
            figure to see its interactive version.}

            \label{fig:ex2_1}
        \end{subfigure}

        \begin{subfigure}{1.0\textwidth}
            \centering
            \href{https://www.gatsby.ucl.ac.uk/~rapela/neuroinformatics/2023/ws1/figures/ex2_distributionSkewed_popmean0.0000_mean0.0000_nSamples10.html}{\includegraphics[width=4in]{../figures/ex2_distributionSkewed_popmean0.0000_mean0.0000_nSamples10.png}}

            \caption{Histogram of p-values of 1.000 randomization tests evaluating if the mean
            of 10 samples from the skewed distribution is equal to zero.
            Click on the figure to see its interactive version.}

            \label{fig:ex2_2}
        \end{subfigure}

        \caption{Exercise 2. The code to generate this figure appears
        \href{https://github.com/joacorapela/neuroinformatics23/blob/master/worksheets/ws1/mySolution/code/scripts/doEx2.py}{here}
        and the parameters used for this script appear
        \href{https://github.com/joacorapela/neuroinformatics23/blob/master/worksheets/ws1/mySolution/code/scripts/doEx2.csh}{here}.
        }

        \label{fig:ex2}

    \end{center}
\end{figure}

\section*{Exercise 3: raster plots}

Please refer to Figure~\ref{fig:ex3}.

\begin{figure}[H]
    \begin{center}

        \begin{subfigure}{1.0\textwidth}
            \centering
            \href{https://www.gatsby.ucl.ac.uk/~rapela/neuroinformatics/2023/ws1/figures/spikes_times_clusterID41_epochedBystimOn_times_sortedByresponse_times_colorschoice.html}{\includegraphics[width=4in]{../figures/spikes_times_clusterID41_epochedBystimOn_times_sortedByresponse_times_colorschoice.png}}

            \caption{Rasterplot of neuron 41 aligned to
            \texttt{stimOn\_times},
            sorted by
            \texttt{response\_times}
            and colored by
            % \texttt{choice}.
            }

            \label{fig:ex3_1}
        \end{subfigure}

        \begin{subfigure}{1.0\textwidth}
            \centering
            \href{https://www.gatsby.ucl.ac.uk/~rapela/neuroinformatics/2023/ws1/figures/spikes_times_clusterID41_epochedBystimOn_times_sortedByresponse_times_colorsfeedbackType.html}{\includegraphics[width=4in]{../figures/spikes_times_clusterID41_epochedBystimOn_times_sortedByresponse_times_colorsfeedbackType.png}}

            \caption{Rasterplot of neuron 41 aligned to
            \texttt{stimOn\_times},
            sorted by
            \texttt{response\_times}
            and colored by
            \texttt{feedbackType}.
            }

            \label{fig:ex3_2}
        \end{subfigure}

        \caption{Exercise 3. The code to generate this figure appears
        \href{https://github.com/joacorapela/neuroinformatics23/blob/master/worksheets/ws1/mySolution/code/scripts/doPlotEpochedSpikesTimes.py}{here}
        and the parameters used for this script appear
        \href{https://github.com/joacorapela/neuroinformatics23/blob/master/worksheets/ws1/mySolution/code/scripts/doPlotEpochedSpikesTimes.csh}{here}.
        }

        \label{fig:ex3}


    \end{center}
\end{figure}

\pagebreak
\begin{appendices}

\section{Under the null hypothesis, p-values are uniformly distributed in $[0,1]$}
\label{sec:pValuesAreUniform}

In Exercise~1, for $i=1,\ldots,1000$, we generated samples from random variables
$\{x_{(i,1)},\ldots,x_{(i,10000)}\}$, from these samples we computed a t-statistic
$t_i=f(x_{(i,1)},\ldots,x_{(i,10000)})$, and from this statistic we calculated a
p-value, $p_i=g(t_i)$. Because the t-statistic is a function, $f$, of random
variables, it can be considered as a random variable, $T$. Because the
p-value is a function, $g$, of a random variables, it can also be considered as
a random variable, $P$. The goal of this section is to prove that the p-value random
variable is uniformly distributed in [0,1]; i.e., $P\sim\mathcal{U}[0,1]$. This
proof is given in Lemma~\ref{lemma:p_values_uniform01}. Before giving this
proof we prove two auxiliary claims (Claims~\ref{claim:pvalue_as_function_of_stat}
and~\ref{claim:uniform_cummulative}).

Figure~\ref{fig:pvalue} illustrates the concept of a p-value. It is the
probability of observing a statistic, $t$, greater than the observed one,
$t_\text{obs}$, when the null hipothesis is true.

\begin{figure}[H]
    \begin{center}
        \includegraphics[width=4in]{../figures/pvalue.png}

        \caption{Illustration of the p-value concept. A p-value is the
        probability of observing a statistic, $t$, greater than the observed
        one, $t_\text{obs}$, when the null hypothesis is true.}
        \label{fig:pvalue}

    \end{center}
\end{figure}

\begin{claim}

    Let $P$ be the p-values random variable, $T$ be the t-statistic random
    variable, and $F_T$ be the cummulative distribution function of $T$. Then
    $P=1-F_T(T)$
    \label{claim:pvalue_as_function_of_stat}

\end{claim}

\begin{proof}
    Let $t_{\text{obs},i}$ and $p_i$ be an observed statistic and associated
    p-value, respectively. Then,

    \begin{align}
        p_i=P(T>t_{\text{obs},i})=1-P(T<t_{\text{obs},i})=1-F_T(t_{\text{obs},i})\label{eq:pvalue_claim_tmp1}
    \end{align}

    The first equality in Eq.~\ref{eq:pvalue_claim_tmp1} follows from the
    definition of a p-value (Fig~\ref{fig:pvalue}). Because
    $p_i=1-F_T(t_{\text{obs},i})$ holds for any pair of samples $p_i$
    and $t_{\text{obs},i}$, then $P=1-F_T(T)$.
\end{proof}

\begin{claim}

    A random variable $U$ is uniformly distributed in [0,1]; i.e.,
    $U\sim\mathcal{U}[0,1]$, if and only if its cummulative distribution
    function is $F_U(u)=P(U<u)=u$, for $u\in[0,1]$.
    \label{claim:uniform_cummulative}

\end{claim}

\begin{proof}

    \begin{align}
        U\sim\mathcal{U}[0,1] &\iff f_U(u)=1 \text{ for } u\in[0,1] \text{ and } f_U(u)=0 \text{ elsewhere } \\
                              &\iff F_U(u)=P(U<u)=\int_0^pf_U(u)dp=u\text{, for }u\in[0,1].
    \end{align}

\end{proof}

\begin{lemma}

    When the null hypothesis holds, p-values are uniformly distributed in
    $[0,1]$; i.e., $P\sim\mathcal{U}[0,1]$.
    \label{lemma:p_values_uniform01}
\end{lemma}

\begin{proof}
    By Claim~\ref{claim:uniform_cummulative} it suffices to show that $F_P(p)=p$.

    \begin{align}
        F_P(p)&=P(P<p)=P(1-F_T(T)<p)=P(1-p<F_T(T)\label{eq:lemmaUniformPvaluesLine1}\\
              &=P(T>F_T^{-1}(1-1))=1-P(T<F_T^{-1}(1-p))\nonumber\\
              &=1-F_T(F_T^{-1}(1-p))=1-(1-p)=p\nonumber
    \end{align}

Note: the second equality in Eq.~\ref{eq:lemmaUniformPvaluesLine1} follows
from Claim~\ref{claim:pvalue_as_function_of_stat}.

\end{proof}

\end{appendices}

\end{document}
