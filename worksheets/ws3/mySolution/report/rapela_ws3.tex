\documentclass[12pt]{article}

\usepackage{natbib}
\usepackage{apalike}
\usepackage[hypertexnames=false,colorlinks=true,breaklinks]{hyperref}
\usepackage{graphicx}
\usepackage[shortlabels]{enumitem}
\usepackage{subcaption}
\usepackage{caption}
\usepackage{float}
\usepackage{amsmath}
\usepackage{amsthm}
\usepackage[title]{appendix}
\usepackage[margin=1in]{geometry}
\usepackage{verbatim}

\newtheorem{definition}{Definition}
\newtheorem{lemma}{Lemma}

\def\fig_width{3.5in}
\title{Report worksheet 3}
\author{Joaqu\'{i}n Rapela}

\begin{document}

\maketitle

\section*{Exercise 1: LFPs}

Figure~\ref{fig:lfpSelectedChannels1} shows the LFP for a few selected channels.


\begin{figure}[H]
    \begin{center}
        \href{https://www.gatsby.ucl.ac.uk/~rapela/neuroinformatics/2023/ws3/figures/lfp_selected_channels_0_16_32_48_64_80_96_112_128_144_160_176_192_208_224_240_256_272_288_304_320_336_352_368_pid_38124fca-a0ac-4b58-8e8b-84a2357850e6.html}{\includegraphics[width=5.5in]{../figures/lfp_selected_channels_0_16_32_48_64_80_96_112_128_144_160_176_192_208_224_240_256_272_288_304_320_336_352_368_pid_38124fca-a0ac-4b58-8e8b-84a2357850e6.png}}
        \caption{LFP of selected channels.
                The script to generate
                this figure appears
                \href{https://github.com/joacorapela/neuroinformatics23/blob/master/worksheets/ws3/mySolution/code/scripts/doPlotSomeChannelsLFPs.py}{here}
                and the parameters used with this script appear
                \href{https://github.com/joacorapela/neuroinformatics23/blob/master/worksheets/ws3/mySolution/code/scripts/doPlotSomeChannelsLFPs.csh}{here}.
                Click on the figure to view its interactive version.
                }

                \label{fig:lfpSelectedChannels1}

            \end{center}
        \end{figure}

\section*{Exercise 2: power spectrum}

Figure~\ref{fig:pxxAllChannels} shows the power spectrum for all
channels.

\begin{figure}[H]
    \begin{center}
        \href{https://www.gatsby.ucl.ac.uk/~rapela/neuroinformatics/2023/ws3/figures/welchPxx_segmentLength_4096_pid_38124fca-a0ac-4b58-8e8b-84a2357850e6.html}{\includegraphics[width=5.5in]{../figures/welchPxx_segmentLength_4096_pid_38124fca-a0ac-4b58-8e8b-84a2357850e6.png}}

        \caption{Power spectrum of all channels, estimated from 600~seconds of
        LFPs.
        We used the Welch method (as
        implemented in the Python function \texttt{scipy.signal.pwelch}) with a
        window of length 4096 samples, or 1.64 seconds, giving a frequency
        resolution of $1/1.64=0.61$~Hz (\texttt{scipy.signal.pwelch} parameter
        \texttt{nperseg}=4096). We generated this figure using
        \href{https://github.com/joacorapela/neuroinformatics23/blob/master/worksheets/ws3/mySolution/code/scripts/doPlotWelchPxxAllChannels.py}{this}
        script, with its default parameters.
        Click on the figure to view its interactive version.  }

                \label{fig:pxxAllChannels}

            \end{center}
        \end{figure}

The formula for the frequency resolution (in Hz) is
$\mathtt{freq\_res\_HZ}=\frac{1}{\mathtt{data\_len\_sec}}$, where
$\mathtt{data\_len\_sec}$ is the length of the data (in seconds) used to
calculate the Fourier transform. Because the Welch method to estimate the power
spectrum computes the Fourier transform using data of the length of its window,
to obtain a frequency resolution less than 1~Hz, we need a window of length
greater than one second, or greater than $sr.fs=2.500$ samples. We used 4,096
samples to estimate the power spectrums in Figure~\ref{fig:pxxAllChannels}.

We see large power around 7~Hz (theta band) in electrodes in the dentate gyrus.
Oscillations in the theta range in the dentate gyrus have been previously
reported~\citep[e.g.,][]{rangelEtAl15}.

We observe stripes of high power at multiples of 50~Hz, that correspond to
contamination of the recordings by the AC power line, which in the UK operates
at 50~Hz. Why do we see high power stripes at multiples of 50~Hz (i.e., 100~Hz
or 150~Hz)? This is because, as proved in
Lemma~\ref{lemma:ftContPeriodicSignal} in the appendix, the Fourier
representation of a periodic continuous signal (like the AC current) is a
scaled set of delta functions at multiples of the frequency of the periodic
signal.

\section*{Exercise 3: spectrogram}

Figure~\ref{fig:spectrogramChannel250} shows the spectrogram for channel 250,
calculated with LFP data of length 1,000~seconds, starting at time 4,500~seconds.`

To overcome the memory problem related to loading long-duration LFP data, we
estimated several spectrograms, with different start times, on windows of
length 1.000 seconds. We only found evidence of power line contamination in the
spectrogram calculated with LFP data starting at time of 4,500~seconds.

\begin{figure}[H]
    \begin{center}
        \href{https://www.gatsby.ucl.ac.uk/~rapela/neuroinformatics/2023/ws3/figures/spectrogram_startTime_4500.00_duration_1000.00_segmentLength_16384_pid_38124fca-a0ac-4b58-8e8b-84a2357850e6_channel_250.html}{\includegraphics[width=5.5in]{../figures/spectrogram_startTime_4500.00_duration_1000.00_segmentLength_16384_pid_38124fca-a0ac-4b58-8e8b-84a2357850e6_channel_250.png}}



        \caption{Spectrogram of channel 250 calculated from LFP data between
        4,500 and 5,500~seconds. Estimation was done
        using the method \texttt{scipy.signal.spectrogram} with a window length
        of \texttt{nperseg}=16,384~samples=6.55~seconds.
        The script to generate this figure appears
        \href{https://github.com/joacorapela/neuroinformatics23/blob/master/worksheets/ws3/mySolution/code/scripts/doComputeAndPlotSpectrogram.py}{here},
        and the parameters used with this script appear
        \href{https://github.com/joacorapela/neuroinformatics23/blob/master/worksheets/ws3/mySolution/code/scripts/doComputeAndPlotSpectrogram.csh}{here}
        Click on the figure to view its interactive version.  }

                \label{fig:spectrogramChannel250}

            \end{center}
        \end{figure}

\section*{Exercise 4: comodugram}

Figure~\ref{fig:comodugramChannel250} shows the comodugram for channel 250.

To overcome the memory problem related to loading long-duration LFP data,
we used a computer with very large RAM memory.

\begin{figure}[H]
    \begin{center}
        \href{https://www.gatsby.ucl.ac.uk/~rapela/neuroinformatics/2023/ws3/figures/comodugram_segmentLength_8192_pid_38124fca-a0ac-4b58-8e8b-84a2357850e6_channel_250_startTime_0.00_duration_5500.00.html}{\includegraphics[width=5.5in]{../figures/comodugram_segmentLength_8192_pid_38124fca-a0ac-4b58-8e8b-84a2357850e6_channel_250_startTime_0.00_duration_5500.00.png}}



        \caption{Comodugram for channel 250. Estimation was done using
        5,500~seconds of LFP data, starting at time zero. We first computed the
        spectrogram using the method \texttt{scipy.signal.spectrogram} with a
        window length of \texttt{nperseg}=8,192 samples=3.28 seconds, and then
        calculated the Pearson correlation coefficient between power at
        different frequencies using function \texttt{np.corrcoef}.  We
        generated this figure with
        \href{https://github.com/joacorapela/neuroinformatics23/blob/master/worksheets/ws3/mySolution/code/scripts/doPlotComodugram.py}{this}
        script using its default parameters.  Click on the figure to view its
        interactive version.  }

                \label{fig:comodugramChannel250}

            \end{center}
        \end{figure}

We see that power at 50~Hz and 100~Hz are highly correlated due to the
harmonics generated by the power line. The blue spots around 8~Hz show that
the theta rhythm in the dentate gyrus appears in a broad set of frequencies
around 8~Hz, and that power at these frequencies is correlated.

\section*{Exercise 5: coherence}

Figure~\ref{fig:coherenceChannel250WithOthers} shows the coherence between
channel 250 and the other channels.

\begin{figure}[H]
    \begin{center}
        \href{https://www.gatsby.ucl.ac.uk/~rapela/neuroinformatics/2023/ws3/figures/coh_startTime_0.00_duration_10.00_segmentLen_-1_pid_38124fca-a0ac-4b58-8e8b-84a2357850e6_channel_250.html}{\includegraphics[width=5.5in]{../figures/coh_startTime_0.00_duration_10.00_segmentLen_-1_pid_38124fca-a0ac-4b58-8e8b-84a2357850e6_channel_250.png}}

        \caption{Coherence between channel 250 and the other channels.
        Estimation was done using 10~seconds of LFP data, starting at time 0,
        with the method \texttt{scipy.signal.coherence}.
        We generated this figure with
        \href{https://github.com/joacorapela/neuroinformatics23/blob/master/worksheets/ws3/mySolution/code/scripts/doPlotCoherence.py}{this}
        script, using its default parameters.
        Click on the figure to view its interactive version.}

                \label{fig:coherenceChannel250WithOthers}

            \end{center}
        \end{figure}

The strip at channel 250 indicates a linear phase relation between the LFP of
this channel and itself at all frequencies. More interestingly, at frequencies
below 150~Hz we observe large coherence (i.e., linear phase relations) limited
to channels in the dentate gyrus molecular layer.

\section*{Exercise 6: cross-spectrum phase}

Figure~\ref{fig:crossSpectrumPhaseChannel250WithOthers} shows the
cross-spectrum phase between
channel 250 and the other channels.

\begin{figure}[H]
    \begin{center}
        \href{https://www.gatsby.ucl.ac.uk/~rapela/neuroinformatics/2023/ws3/figures/csdAngle_startTime_0.00_duration_600.00_segmentLen_-1_pid_38124fca-a0ac-4b58-8e8b-84a2357850e6_channel_250.html}{\includegraphics[width=5.5in]{../figures/csdAngle_startTime_0.00_duration_600.00_segmentLen_-1_pid_38124fca-a0ac-4b58-8e8b-84a2357850e6_channel_250.png}}

        \caption{Cross-spectrum phase between channel 250 and the other
        channels. Estimation was done using the method
        \texttt{scipy.signal.csd} in a time window of 600 seconds starting at
        time zero.  We generated this figure with 
        \href{https://github.com/joacorapela/neuroinformatics23/blob/master/worksheets/ws3/mySolution/code/scripts/doPlotCrossSpectrumPhase.py}{this}
        script, using its default parameters.
        Click on the figure to view its interactive version.}

        \label{fig:crossSpectrumPhaseChannel250WithOthers}

        \end{center}
\end{figure}

Figure~\ref{fig:crossSpectrumPhaseChannel250WithOthers} shows that channel 250
is approximately 180 degrees out of phase with channels 200 and 300, and
approximately in phase with channel 170.

Figure~\ref{fig:pxxAllChannels} shows that the largest power at channel 200
($0.003~mV^2$) is comparable to that at channel 250 ($0.008~mV^2$), but an
order of magnitude larger than the maximum power at channel 300.
($0.0005~mV^2$). Thus, the phase difference with channel 250 should be more
apparent for channel 200 than for channel 300, as illustrated in
Figure~\ref{fig:lfpSelectedChannels2}.
%
Also the largest power at channel 170 ($0.001~mV^2$) is intermediate with that
of channels 200 and 300. Thus, the phase relation of channel 250 with channel 170
should be less apparent than that with channel 200, but more apparent than that
with channel 300, as also shown in Figure~\ref{fig:lfpSelectedChannels2}.

\begin{figure}[H]
    \begin{center}
        \href{https://www.gatsby.ucl.ac.uk/~rapela/neuroinformatics/2023/ws3/figures/lfp_selected_channels_170_200_250_300_pid_38124fca-a0ac-4b58-8e8b-84a2357850e6.html}{\includegraphics[width=5.5in]{../figures/lfp_selected_channels_170_200_250_300_pid_38124fca-a0ac-4b58-8e8b-84a2357850e6.png}}

        \caption{LFP of channels 170, 200, 250 and 300. These LFPs agree with
        Figure~\ref{fig:crossSpectrumPhaseChannel250WithOthers} in that the LFP
        at channel 250 has opposite phase than that at channel 200 and 300, but
        similar phase than the LFP at channel 170.  The script used to generate
        this figure appears
        \href{https://github.com/joacorapela/neuroinformatics23/blob/master/worksheets/ws3/mySolution/code/scripts/doPlotSomeChannelsLFPs.py}{here}
        and the parameters used with this script appear
        \href{https://github.com/joacorapela/neuroinformatics23/blob/master/worksheets/ws3/mySolution/code/scripts/doPlotSomeChannelsLFPs.csh}{here}.
        Click on the figure to view its interactive version.}

        \label{fig:lfpSelectedChannels2}

        \end{center}
\end{figure}

\pagebreak
\begin{appendices}

\section{Fourier transform of a continuous periodic signal}
\label{sec:ftContPeriodicSignal}

    We prove in Lemma~\ref{lemma:ftContPeriodicSignal} that the Fourier
    transform of a continuous periodic signal is a sum of scaled delta
    functions at multiples of the frequency of this signal.

    \begin{definition}[continuous signal]

        A continuous signal $x(t)$ is periodic if and only if there exists
        a period $T>0$ such that

        \begin{align}
            x(t)=x(t+T), \forall t\in\Re
        \end{align}

        \label{definition:periodicSignal}
    \end{definition}

    \begin{lemma}[Fourier transform of a periodic signal]
        If $x(t)$ is periodic, with period $T$, then

        \begin{align}
            \mathcal{FT}\{x(t)\}(j\Omega)=2\pi\sum_{k=-\infty}^\infty X^S[k]\;\delta\left(\Omega-\frac{2\pi k}{T}\right)\label{eq:ftPeriodic}
         \end{align}

         \noindent with $X^S[k]$ the Fourier series coefficient at frequency $k$ (Eq.~\ref{eq:fsCoefficient}).

         \label{lemma:ftContPeriodicSignal}

    \end{lemma}

    \begin{proof}
        Because $x(t)$ is a periodic signal, it admits a Fourier series
        representation \citep[][Section 2.3]{porat97}

        \begin{align}
            x(t)&=\sum_{k=-\infty}^\infty X^S[k]\exp\left(\frac{j2\pi kt}{T}\right)\label{eq:xFourierSeries}\\
            \text{with}&\nonumber\\
            X^S[k]&=\frac{1}{T}\int_{-T/2}^{T/2}x(t)\exp\left(-\frac{j2\pi kt}{T}\right)\label{eq:fsCoefficient}
        \end{align}

        By the linearity of the Fourier transform \citep[][Eq. 2.4]{porat97}, from Eq.~\ref{eq:xFourierSeries},
        we have

        \begin{align}
            \mathcal{FT}\{x(t)\}(j\Omega)&=\sum_{k=-\infty}^\infty
            X^S[k]\;\mathcal{FT}\left\{\exp\left(\frac{j2\pi
            kt}{T}\right)\right\}(j\Omega)\label{eq:xFT}
        \end{align}

        We next compute the Fourier transform of the exponential in the right hand
        side of Eq.~\ref{eq:xFT}

        \begin{align}
            \mathcal{FT}\left\{\exp\left(\frac{j2\pi kt}{T}\right)\right\}(j\Omega)&=\mathcal{FT}\left\{1\; \exp\left(\frac{j2\pi kt}{T}\right)\right\}(j\Omega)\label{eq:ftExp1}\\
                                          &=\mathcal{FT}\left\{1\right\}\left(j\left(\Omega-\frac{2\pi k}{T}\right)\right)\label{eq:ftExp2}\\
                                          &=2\pi\;\delta\left(\Omega-\frac{2\pi k}{T}\right)\label{eq:ftExp3}
        \end{align}

        Notes:
        \begin{enumerate}

            \item Eq.~\ref{eq:ftExp2} follows from Eq.~\ref{eq:ftExp1} by the
                the frequency shift property of the Fourier
                transform\footnote{$y(t)=e^{j\Omega_0 t}x(t)\leftrightarrow
                Y(j\Omega)=X\left(j(\Omega-\Omega_0)\right)$, \citet[][Section
                2.1]{porat97}}.

            \item Eq.~\ref{eq:ftExp3} follows from Eq.~\ref{eq:ftExp2} by the
                Fourier transform of the DC function (Lemma~\ref{lemma:ftDC}).

        \end{enumerate}

        Replacing Eq.~\ref{eq:ftExp3} into Eq.~\ref{eq:xFT}
        yields Eq.~\ref{eq:ftPeriodic}.

    \end{proof}

    \begin{lemma}[Fourier transform of the DC function]
        \begin{align}
            \mathcal{FT}\{1\}(j\Omega)=2\pi\;\delta(\Omega)
        \end{align}
        \label{lemma:ftDC}
    \end{lemma}

    \begin{proof}
        We start by computing the Fourier transform of the delta function.

        \begin{align}
            \mathcal{FT}\{\delta(t)\}(j\Omega)=\int_{-\infty}^\infty\delta(t)\exp\left(-j\Omega t\right)\ dt=\left.\exp\left(-j\Omega t\right)\right |_{t=0}=1
        \end{align}

        Then by the duality property of the Fourier transform
        (Lemma~\ref{lemma:ftDuality}) we have

        \begin{align}
            \mathcal{FT}\{1\}(j\Omega)=2\pi\;\delta(-\Omega)=2\pi\;\delta(\Omega)\label{eq:ftDC1}
        \end{align}

        Notes:
        \begin{enumerate}

            \item The last equality in Eq.~\ref{eq:ftDC1} holds because the
                delta function is even.

        \end{enumerate}
    \end{proof}

    \begin{lemma}[Duality of the Fourier transform]

        Let $x(t)$ be a signal and $X(j\Omega)$ be its Fourier transform, then

        \begin{align}
            \mathcal{FT}\{X(jt)\}(j\Omega)=2\pi\;x(-\Omega)
        \end{align}

        \label{lemma:ftDuality}
    \end{lemma}

    \begin{proof}
        If $x(t)$ is a signal, with real or complex values, and
        $X(j\Omega)$ is its Fourier transform, then they are related by the
        following equations \citep[][Section 2.1]{porat97}

        \begin{align}
            X(j\Omega)=\mathcal{FT}\{x(t)\}(j\Omega)&=\int_{-\infty}^\infty x(t)\exp\left(-j\Omega t\right)dt\label{eq:ft}\\
            x(t)=\mathcal{IFT}\{X(j\Omega)\}(t)&=\frac{1}{2\pi}\int_{-\infty}^\infty X(j\Omega)\exp\left(j\Omega t\right)d\Omega\label{eq:ift}
        \end{align}

        Then

        \begin{align}
            \mathcal{FT}\left\{X(jt)\right\}(j\Omega)&=\int_{-\infty}^\infty X(jt)\exp\left(-j\Omega t\right)dt\label{eq:dualityFT1}\\
                                                     &=2\pi\left(\frac{1}{2\pi}\int_{-\infty}^\infty X(jt)\exp\left(jt(-\Omega)\right)dt\right)\label{eq:dualityFT2}\\
                                                     &=2\pi\;x(-\Omega)\label{eq:dualityFT3}
        \end{align}

        Notes:

        \begin{enumerate}

            \item in Eq.~\ref{eq:dualityFT1} we applied the Fourier transform (Eq.~\ref{eq:ft}) to the complex signal $X(jt)$

            \item in Eq.~\ref{eq:dualityFT3} we used the inverse Fourier
                transform (Eq.~\ref{eq:ift}) with the change of variables
                $\Omega$ in Eq.~\ref{eq:ift} to $t$ in Eq.~\ref{eq:dualityFT2}
                and $t$ in Eq.~\ref{eq:ift} to $-\Omega$ in
                Eq.~\ref{eq:dualityFT2}.

        \end{enumerate}
    \end{proof}

\end{appendices}

\bibliographystyle{apalike}
\bibliography{others,signalProcessing}

\end{document}
