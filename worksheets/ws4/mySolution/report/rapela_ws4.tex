\documentclass[12pt]{article}

\usepackage{natbib}
\usepackage{apalike}
\usepackage[hypertexnames=false,colorlinks=true,breaklinks]{hyperref}
\usepackage{graphicx}
\usepackage[shortlabels]{enumitem}
\usepackage{subcaption}
\usepackage{caption}
\usepackage{float}
\usepackage{amsmath}
\usepackage{amsthm}
\usepackage[title]{appendix}
\usepackage[margin=1in]{geometry}
\usepackage{verbatim}

\newtheorem{definition}{Definition}
\newtheorem{lemma}{Lemma}

\def\fig_width{3.5in}
\title{Report worksheet 4}
\author{Joaqu\'{i}n Rapela}

\begin{document}

\maketitle

\section*{Exercise 1: Permutation test using locally-weighted log likelihood}

I used a sum of the locally-weighted log likelihood function, \texttt{lwll},
function values as the test statistic. This function takes three parameters:
$x, \theta_x, \kappa_x$. I considered ten equally-spaced values of $x_i$ in [0,
1]. For each $x_i$ I took the angles $\{\theta_j\}$ of the $x_j$'s in the
neighbourhood of $x_i$ (i.e., of the $x_j$s such that $|x_i-x_j|<0.005$). I
passed these angles as inputs to the function \texttt{scipy.stats.vonmises.fit}
to estimate $\theta_{x_i}$ and $\kappa_{x_i}$. The test statistic
\texttt{test\_lwll} I used is:

\begin{align}
    % \text{test_stat}(\{x_t,\theta_t\})=\sum_{i=1}^{10}\text{lwll}\left(\{x_t,\theta_t\}|x_i,\theta_{x_i},\kappa_{x_i}\right)
    \text{lwll\_stat}(\{x_t,\theta_t\})=\sum_{i=1}^{10}\text{lwll}\left(\{x_t,\theta_t\}|x_i,\theta_{x_i},\kappa_{x_i}\right)
\end{align}

Given a dataset $\{x_i, \theta_i\}, i=1,\ldots,N$, I employed as shuffled dataset
$\{x_i, \theta_{p(i)}\}$, with $\{p(1),\ldots,p(N)\}$ a permutation of
$\{1,\ldots,N\}$.

A Python implementation of \texttt{lwll\_stat} and of \texttt{lwll} can be
found
\href{https://github.com/joacorapela/neuroinformatics23/blob/master/worksheets/ws4/mySolution/code/scripts/utils.py}{here}
and a script calling these functions to perform permutation tests can be found
\href{https://github.com/joacorapela/neuroinformatics23/blob/master/worksheets/ws4/mySolution/code/scripts/doPermutationTestLWLL.py}{here}.
To improve runtime, this script uses multiprocessing.

Figure~\ref{fig:data_non_independent} plots the non-independent 
dataset in the exercise statement and
Figure~\ref{fig:perm_test_non_independent_data} shows the results of a
permutation test, with 1000 resamples, on this dataset. The independence null
hypothesis was rejected with a p-value of zero.

\begin{figure}
    \begin{center}
        \href{https://www.gatsby.ucl.ac.uk/~rapela/neuroinformatics/2023/ws4/figures/data_nSamples_1000_kappa_1.00_locIntercept_0.00_locSlope_6.283185.html}{\includegraphics[width=4.0in]{../figures/data_nSamples_1000_kappa_1.00_locIntercept_0.00_locSlope_6.283185.png}}

        \caption{Non-independent dataset in the exercise statement used to
        calculate the permutation test in
        Figure~\ref{fig:perm_test_non_independent_data}. To generate this
        figure, I used
        \href{https://github.com/joacorapela/neuroinformatics23/blob/master/worksheets/ws4/mySolution/code/scripts/doPlotSampledData.py}{this}
        script, with its default parameters.  Click on the image to view its
        interactive version.}

        \label{fig:data_non_independent}

    \end{center}
\end{figure}

\begin{figure}
    \begin{center}
        \href{https://www.gatsby.ucl.ac.uk/~rapela/neuroinformatics/2023/ws4/figures/lwllPermutationTest_nSamples_1000_kappa_1.00_locIntercept_0.00_locSlope_6.283185_nXs_10_nResamples_1000_nBins_20.html}{\includegraphics[width=4.0in]{../figures/lwllPermutationTest_nSamples_1000_kappa_1.00_locIntercept_0.00_locSlope_6.283185_nXs_10_nResamples_1000_nBins_20.png}}

        \caption{Result from a permutation test for the non-independent dataset
        in the exercise statement (Figure~\ref{fig:data_non_independent}). I
        used 1000 resamples. The independence null hypothesis was rejected with
        a p-value of zero. I performed this test using
        \href{https://github.com/joacorapela/neuroinformatics23/blob/master/worksheets/ws4/mySolution/code/scripts/doPermutationTestLWLL.py}{this}
        script with its default parameters. I generated this figures using
        \href{https://github.com/joacorapela/neuroinformatics23/blob/master/worksheets/ws4/mySolution/code/scripts/doPlotPermutationTestResult.py}{this}
        script, with its default parameters. Click on the image to view its
        interactive version.}

        \label{fig:perm_test_non_independent_data}

    \end{center}
\end{figure}

Figure~\ref{fig:data_independent} plots an independent dataset, where all
samples of $\theta$ were generated from a von Mises distribution with
parameters $\theta=0, \kappa=1$.  Figure~\ref{fig:perm_test_independent_data}
shows the results of a permutation test, with 1000 resamples, on this dataset.
The independence null hypothesis was not rejected with p-value=0.427

\begin{figure}
    \begin{center}
        \href{https://www.gatsby.ucl.ac.uk/~rapela/neuroinformatics/2023/ws4/figures/data_nSamples_1000_kappa_1.00_locIntercept_0.00_locSlope_0.000000.html}{\includegraphics[width=4.0in]{../figures/data_nSamples_1000_kappa_1.00_locIntercept_0.00_locSlope_0.000000.png}}

        \caption{Independent dataset used to calculate the permutation test in
        Figure~\ref{fig:perm_test_independent_data}.  To generate this figure,
        I used
        \href{https://github.com/joacorapela/neuroinformatics23/blob/master/worksheets/ws4/mySolution/code/scripts/doPlotSampledData.py}{this}
        script, with its default parameters, except \texttt{--loc\_slope=0}.
        Click on the image to view its interactive version.}

        \label{fig:data_independent}

    \end{center}
\end{figure}

\begin{figure}
    \begin{center}
        \href{https://www.gatsby.ucl.ac.uk/~rapela/neuroinformatics/2023/ws4/figures/lwllPermutationTest_nSamples_1000_kappa_1.00_locIntercept_0.00_locSlope_0.000000_nXs_10_nResamples_1000_nBins_20.html}{\includegraphics[width=4.0in]{../figures/lwllPermutationTest_nSamples_1000_kappa_1.00_locIntercept_0.00_locSlope_0.000000_nXs_10_nResamples_1000_nBins_20.png}}

        \caption{Result from a permutation test for the independent dataset
        (Figure~\ref{fig:data_independent}).  I
        used 1000 resamples. The independence null hypothesis was not rejected
        with a p-value of 0.427. I performed this test using
        \href{https://github.com/joacorapela/neuroinformatics23/blob/master/worksheets/ws4/mySolution/code/scripts/doPermutationTestLWLL.py}{this}
        script, with its default parameters, except by \texttt{--loc\_slope=0}.
        I generated this figures using
        \href{https://github.com/joacorapela/neuroinformatics23/blob/master/worksheets/ws4/mySolution/code/scripts/doPlotPermutationTestResult.py}{this}
        script with its default parameters, except by \texttt{--loc\_slope=0}.
        Click on the image to view its interactive version.}

        \label{fig:perm_test_independent_data}

    \end{center}
\end{figure}


\end{document}
