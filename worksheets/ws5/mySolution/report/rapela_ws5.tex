\documentclass[12pt]{article}

\usepackage{natbib}
\usepackage{apalike}
\usepackage[hypertexnames=false,colorlinks=true,breaklinks]{hyperref}
\usepackage{graphicx}
\usepackage[shortlabels]{enumitem}
\usepackage{subcaption}
\usepackage{caption}
\usepackage{float}
\usepackage{amsmath}
\usepackage{amssymb}
\usepackage{amsthm}
\usepackage[title]{appendix}
\usepackage[margin=1in]{geometry}
\usepackage{verbatim}
\usepackage[many]{tcolorbox}

\newtheorem{definition}{Definition}
\newtheorem{lemma}{Lemma}
\newtheorem{theorem}{Theorem}

\tcolorboxenvironment{theorem}{
    colback=blue!5!white,
    boxrule=0pt,
    boxsep=1pt,
    left=2pt,right=2pt,top=2pt,bottom=2pt,
    oversize=2pt,
    sharp corners,
    before skip=\topsep,
    after skip=\topsep,
}

\tcolorboxenvironment{definition}{
    colback=blue!5!white,
    boxrule=0pt,
    boxsep=1pt,
    left=2pt,right=2pt,top=2pt,bottom=2pt,
    oversize=2pt,
    sharp corners,
    before skip=\topsep,
    after skip=\topsep,
}

\tcolorboxenvironment{lemma}{
    colback=blue!5!white,
    boxrule=0pt,
    boxsep=1pt,
    left=2pt,right=2pt,top=2pt,bottom=2pt,
    oversize=2pt,
    sharp corners,
    before skip=\topsep,
    after skip=\topsep,
}

\def\fig_width{3.5in}
\title{Report worksheet 5}
\author{Joaqu\'{i}n Rapela}

\begin{document}

\maketitle

\section*{Exercise 1: z-scored binned spikes}

Figure~\ref{fig:zscores_unsorted} shows the zscores of the binned spikes of all
neurons (bin size=1~sec, unsorted neurons). I choosed \texttt{zmin} and
\texttt{zmax} as the 1\% and 99\% percentiles of the zscores distribution,
because, as shown below, the negative z-values are informative.


\begin{figure}[H]
    \begin{center}
        \href{https://www.gatsby.ucl.ac.uk/~rapela/neuroinformatics/2023/ws5/figures/binned_spikes_binSize_1.00_original.html}{\includegraphics[width=5.5in]{../figures/binned_spikes_binSize_1.00_original.png}}

        \caption{z-scores of binned spikes times of all neurons (bin
        size=1~sec, unsorted neurons).  Generated with
        \href{https://github.com/joacorapela/neuroinformatics23/blob/master/worksheets/ws5/mySolution/code/scripts/doEx1Plotly.py}{this}
        script using its default parameters. Click on the image to see its
        interactive version.}

        \label{fig:zscores_unsorted}
    \end{center}
\end{figure}

If I don't limit the \texttt{zmax} of the heatmap colors become imbalanced due
to a neuron with high firing at one time
(Figure~\ref{fig:zscores_unsorted_noZmax}). 

\begin{figure}[H]
    \begin{center}
        \href{https://www.gatsby.ucl.ac.uk/~rapela/neuroinformatics/2023/ws5/figures/binned_spikes_binSize_1.00_noZmax.html}{\includegraphics[width=5.5in]{../figures/binned_spikes_binSize_1.00_noZmax.png}}

        \caption{z-scores of binned spikes times of all neurons, plotted
        without \texttt{zmax} (bin size=1~sec, unsorted neurons). Generated
        with
        \href{https://github.com/joacorapela/neuroinformatics23/blob/master/worksheets/ws5/mySolution/code/scripts/doEx1Plotly.py}{this}
        script using its default parameters. Click on the image to see its
        interactive version.}

        \label{fig:zscores_unsorted_noZmax}
    \end{center}
\end{figure}

If I don't z-score the binned spikes colors become imbalanced due to neurons that
have large mean firing rate (Figure~\ref{fig:zscores_unsorted_noZscored}).

\begin{figure}[H]
    \begin{center}
        \href{https://www.gatsby.ucl.ac.uk/~rapela/neuroinformatics/2023/ws5/figures/binned_spikes_binSize_1.00_noZscored.html}{\includegraphics[width=5.5in]{../figures/binned_spikes_binSize_1.00_noZscored.png}}

        \caption{Non-zscored binned spikes times of all neurons (bin
        size=1~sec, unsorted neurons).  Generated with
        \href{https://github.com/joacorapela/neuroinformatics23/blob/master/worksheets/ws5/mySolution/code/scripts/doEx1Plotly.py}{this}
        script using its default parameters. Click on the image to see its
        interactive version.}

        \label{fig:zscores_unsorted_noZscored}
    \end{center}
\end{figure}

\section*{Exercise 2: application of the SVD to z-scored binned spikes}

Figure~\ref{fig:zscores_vh0Sorted} plots the same z-scored binned spikes of
Figure~\ref{fig:zscores_unsorted}, but with neurons ordered according to their
weight along the first right singular vector.


\begin{figure}[H]
    \begin{center}
        \href{https://www.gatsby.ucl.ac.uk/~rapela/neuroinformatics/2023/ws5/figures/binned_spikes_svd_binSize_1.00_vh0Sorted.html}{\includegraphics[width=5.5in]{../figures/binned_spikes_svd_binSize_1.00_vh0Sorted.png}}

        \caption{Same as Figure~\ref{fig:zscores_unsorted}, but neurons have
        been sorted according to their weight along the first left singular
        vector. The black vertical line indicates the last response time of the
        subject. Generated with
        \href{https://github.com/joacorapela/neuroinformatics23/blob/master/worksheets/ws5/mySolution/code/scripts/doEx2Plotly.py}{this}
        script using its default parameters. Click on the image to see its
        interactive version.}

        \label{fig:zscores_vh0Sorted}
    \end{center}
\end{figure}

The first left right singular vector (scaled by the corresponding entry in the
first right singular vector) gives the temporal profile of the best rank-one approximation of the
binned spikes of any neuron.
%
Figure~\ref{fig:firstLeftSingularVector} plots in blue a part of the the first
left singular vector between 400 and 650~seconds. The block vertical lines
indicate subject response times. Interestingly, we see that this approximation
of the binned spikes times tends to peak immediately after the subject response
times. This suggest a synchronization between neurons' spikes and subject's
responses.

\begin{figure}[H]
    \begin{center}
        \href{https://www.gatsby.ucl.ac.uk/~rapela/neuroinformatics/2023/ws5/figures/binned_spikes_svd_binSize_1.00_uFirstCol.html}{\includegraphics[width=5.5in]{../figures/binned_spikes_svd_binSize_1.00_uFirstCol_segment.png}}

        \caption{Part of the first left singular vector (blue trace) and subject response times
        (black vertical lines). This figure suggests a synchronization betweeen
        neurons' spikes and subjects responses (see text).
        Generated with
        \href{https://github.com/joacorapela/neuroinformatics23/blob/master/worksheets/ws5/mySolution/code/scripts/doEx2Plotly.py}{this}
        script using its default parameters. Click on the image to see its
        interactive version.}

        \label{fig:firstLeftSingularVector}
    \end{center}
\end{figure}


The nth entry of the first right singular vector gives us the weight of the
first left singular vector to approximate the z-scores of the binned spikes of
neuron $n$.  Figure~\ref{fig:histEntriesFirstRightSingularVector} plots the
histogram of entries of the first right singular vector. We see
weights as positive as 0.13, corresponding to neurons with z-scored binned
spikes correlated to the first left singular vector. We also see weights
as negative as -0.12, corresponding to neurons with z-scored binned spikes anti
correlated to the first left singular vector.

\begin{figure}[H]
    \begin{center}
        \href{https://www.gatsby.ucl.ac.uk/~rapela/neuroinformatics/2023/ws5/figures/binned_spikes_svd_binSize_1.00_uFirstColWeight.html}{\includegraphics[width=5.5in]{../figures/binned_spikes_svd_binSize_1.00_uFirstColWeight.png}}

        \caption{Histogram of entries in the first right left singular vector.
        Generated with
        \href{https://github.com/joacorapela/neuroinformatics23/blob/master/worksheets/ws5/mySolution/code/scripts/doEx2Plotly.py}{this}
        script using its default parameters. Click on the image to see its
        interactive version.}

        \label{fig:histEntriesFirstRightSingularVector}
    \end{center}
\end{figure}

The weight of the first left singular vector on the z-scored binned spikes of
the neurons near the top of Figure~\ref{fig:zscores_vh0Sorted} is large and
positive. We see vertical red stripes on the z-scores of these neurons,
indicating their strong synchronization with the response times of the subject.
%
Before the time of the last response of the subject (black vertical line)
z-scores tend to be positive (i.e., binned spikes larger than their mean), but
after the last response of the subject z-scores tend to be negative.
%
Moving the computer cursor over the top of the figure shows that most of these
neurons belong to the primary motor cortex\footnote{areas: MOp5 (layer 5),
MOp6a (layer 6a), MOp6b (layer 6b)} and striatum\footnote{areas: CP
(caudoputamen), STR (striatum)}.

The oposite happens to neurons near the bottom of
Figure~\ref{fig:zscores_vh0Sorted}. For these neurons the weight of the first
left singular vector on their z-scored binned spikes is large and negative. We
don't see vertical stripes on their z-scores. Before the time of the last
response of the subject (black vertical line) z-scores tend to be negative
(i.e., binned spikes lower than their mean), but after the last response of the
subject z-scores tend to be positive.
%
Moving the computer cursor over the top of the figure shows that most neurons
belong to the pallidum\footnote{area BST (bed nuclei of the stria terminalis)}.

\pagebreak
\begin{appendices}

\section{Notes on the SVD}
\label{sec:notesOnTheSVD}

    \begin{definition}[Rank of a matrix]

        The column rank of a matrix is the dimension of the space spanned by
        its columns. Similarly, the row rank of a matrix is the dimension of
        the space spanned by its rows. The column rank of a matrix is always
        equal to its row rank. This is a corollary of the SVD. So we refer to
        this number simply as the rank of a matrix.

        \label{def:rank}
    \end{definition}

    The rank of a matrix can be interpreted as a measure of the complexity of
    the matrix. Matrices with lower rank are simpler than those with larger
    rank.

    The SVD decomposes a matrix as a sum of rank-one (i.e., very simple)
    matrices. 

    \begin{align}
        M = \sum_{k=1}^rs_k\mathbf{u}_k\mathbf{v}_k^*
        \label{eq:svd}
    \end{align}

    There are multiple other decompositions as sums of rank-one matrices. If
    $M\in\mathbb{C}^{m\times n}$, then it can be decomposed as a sum of $m$
    rank-one matrices given by its rows (i.e.,
    $M=\sum_{i=1}^m\mathbf{e}_i\mathbf{m}_{i,\cdot}^*$, where $\mathbf{e}_i$ is
    the m-dimensional canonical unit vector, and $\mathbf{m}_{i,\cdot}$ is the ith row
    of $M$), or as a sum of $n$ rank-one matrices given by its columns (i.e.,
    $M=\sum_{j=1}^n\mathbf{m}_{\cdot,j}\mathbf{e}_j^*$, where $\mathbf{e}_j$ is
    the n-dimensional canonical unit vector, and $\mathbf{m}_{\cdot,j}$ is the jth
    column of $M$), or a sum of $mn$ rank-one matrices each containing only one
    non-zero element (i.e., $M=\sum_{i=1}^m\sum_{j=1}^nm_{ij}E_{ij}$, where
    $E_{ij}$ is the matrix with all entries equal to zero, except the $ij$
    entry that is one, and $m_{ij}$ is the entry of $M$ at position ij).

    A unique characteristic of the SVD compared to these other decompositions
    is that, if the rank of a matrix is $r$, then its SVD yields optimal
    approximations of lower rank $\nu$, for $\nu=1,\ldots,r$, as shown by
    Theorem~\ref{thm:eckart-young-mirsky}.

    \begin{definition}[Frobenius norm]
        The Frobenius norm of matrix $M\in\mathbb{C}^{m\times n}$ is

        \begin{align}
            \|M\|_F=\left(\sum_{i=1}^m\sum_{j=1}^nm_{ij}^2\right)^{1/2}
        \end{align}
        \label{def:frobeniusNorm}

    \end{definition}

    Note that

    \begin{align}
        \|M\|_F=\sqrt{tr(M^*M)}=\sqrt{tr(MM^*)}
        \label{eq:frobeniusAsTrace}
    \end{align}

    \begin{lemma}[Orthogonal matrices preserve the Frobenius norm]
        Let $M\in\mathbb{C}^{m\times n}$ and let $P\in\mathbb{C}^{m\times m}$
        and $Q\in\mathbb{C}^{n\times n}$ be orthogonal matrices. Then

        \begin{align}
            \|PMQ\|_F=\|M\|_F
        \end{align}

        \label{lemma:orthogonalPreserveF}
    \end{lemma}

    \begin{proof}
        \begin{align}
            \|PMQ\|_F&=\sqrt{tr((PMQ)(PMQ)^*)}=\sqrt{tr(PMQQ^*M^*P^*)}=\sqrt{tr(PMM^*P^*)\label{eq:frobInvLine1}}\\
                     &=\sqrt{tr(P^*PMM^*)}=\sqrt{tr(MM^*)}=\|M\|_F\label{eq:frobInvLine2}
        \end{align}
        Notes:
        \begin{enumerate}
            \item The first equality in Eq.~\ref{eq:frobInvLine1} follows
                Eq.~\ref{eq:frobeniusAsTrace}.
            \item The second equality in Eq.~\ref{eq:frobInvLine1} uses the fact
                that $(AB)^*=B^*A^*$.
            \item The third equality in Eq.~\ref{eq:frobInvLine1} holds because
                $Q$ is orthogonal (i.e., $QQ^*=I$).
            \item The first equality in Eq.~\ref{eq:frobInvLine2} uses the
                ciclic property of the trace (i.e., tr(ABC)=tr(CAB)).
            \item The first equality in Eq.~\ref{eq:frobInvLine2} holds by the
                orthogonality of $P$.
            \item The last equality in Eq.~\ref{eq:frobInvLine2} again applies
                Eq.~\ref{eq:frobeniusAsTrace}.
        \end{enumerate}
    \end{proof}

    A direct consequence of Lemma~\ref{lemma:orthogonalPreserveF} is that the
    Frobenius norm of any matrix $M=USV^*$ is

    \begin{align}
        \|M\|_F=\|USV^*\|_F=\|S\|_F=\sqrt{\sum_{k=1}^rs_k^2}
    \end{align}

    Another consequence of Lemma~\ref{lemma:orthogonalPreserveF} is 
    the error in approximating a matrix $M$ of rank $r$ with its truncated
    SVD of rank $\nu$ (i.e., $M_\nu=\sum_{k=1}^\nu s_k\mathbf{u}_k\mathbf{v}_k^*$) is

    \begin{align}
        \|M-M_\nu\|_F=\|\sum_{k=1}^rs_k\mathbf{u}_k\mathbf{v}_k^*-\sum_{k=1}^\nu
        s_k\mathbf{u}_k\mathbf{v}_k^*\|_F=\|\sum_{k={\nu+1}}^rs_k\mathbf{u}_k\mathbf{v}_k^*\|_F=\sqrt{\sum_{k=\nu+1}^rs_k^2}\label{eq:truncSVDerror}
    \end{align}

    \begin{theorem}[Eckart-Young-Mirsky]
        Let $M\in\mathbb{C}^{m\times n}$ be of rank r with singular value
        decomposition $M=\sum_{k=1}^rs_k\mathbf{u}_k\mathbf{v}_k^*$. For
        any $\nu$ with $0\leq\nu\leq r$, define

        
        \begin{align}
            M_\nu=\sum_{k=1}^\nu s_k\mathbf{u}_k\mathbf{v}_k^*
        \end{align}

        Then

        \begin{align}
            \|M-M_\nu\|_F=\inf_{\substack{\tilde{M}\in\mathbb{C}^{m\times n}\\\text{rank}(\tilde{M})\leq\nu}}\|M-\tilde{M}\|_F=\sqrt{\sum_{k=\nu+1}^rs_k^2}\label{eq:errorFNorm}
        \end{align}

        \label{thm:eckart-young-mirsky}
    \end{theorem}

    \begin{proof}
        We use the Weyl's inequality that relates the singular values of a sum
        of two matrices to the singular values of each of these matrices.
        Precisely, if $X,Y\in\mathbb{C}^{m\times n}$ and $s_i(X)$ is the ith
        singular value of $X$, then

        \begin{align}
            s_{i+j-1}(X+Y)\leq s_i(X)+s_j(Y)
            \label{eq:weylsInequality}
        \end{align}

        Let $\tilde{M}$ be a matrix of rank at most $\nu$. Applying
        Eq.~\ref{eq:weylsInequality} to $X=M-\tilde{M}$, $Y=\tilde{M}$ and
        $j-1=\nu$ we obtain

        \begin{align}
            s_{i+\nu}(M)\leq s_i(M-\tilde{M})+s_{\nu+1}(\tilde{M})=s_i(M-\tilde{M})\label{eq:svMandMerror}
        \end{align}

        The last equality in Eq.~\ref{eq:svMandMerror} holds because $\tilde{M}$
        has rank less or equal to $\nu$, and therefore its $\nu+1$ singular value is zero.

        \begin{align}
            \|M-M_\nu\|_F^2&=\sum_{j=\nu+1}^rs_j^2(M)=\sum_{i=1}^{r-\nu}s_{i+\nu}^2(M)\leq\sum_{i=1}^{r-\nu}s_i^2(M-\tilde{M})\leq\sum_{i=1}^{\min(m,n)}s_i^2(M-\tilde{M})\label{eq:final1}\\
                           &=\|M-\tilde{M}\|_F^2\label{eq:final2}
        \end{align}

        Notes:
        \begin{enumerate}
            \item The first equality in Eq.~\ref{eq:final1} holds by
                Eq.~\ref{eq:errorFNorm}.
            \item The second equality in Eq.~\ref{eq:final1} used the change of
                variables $i=j-\nu$.
            \item The first inequality in Eq.~\ref{eq:final1} used
                Eq.~\ref{eq:svMandMerror}
            \item The last inequality in Eq.~\ref{eq:final1} is true because
                $r-\nu<=\min(m,n)$ and adding squared eigenvalues to the sum in
                the left hand side increases this sum.
            \item The equality in Eq.~\ref{eq:final2} again holds by
                Eq.~\ref{eq:errorFNorm}.
        \end{enumerate}

        The last equality in Eq.~\ref{eq:errorFNorm} follows from
        Eq.~\ref{eq:truncSVDerror}.

    \end{proof}

\section{Verification of the Eckart-Young-Mirsky theorem}
\label{sec:verificationEckartYoungMirsky}

To verify Theorem~\ref{thm:eckartYoungMirsky}, for different ranks $\nu$, I
will compute truncated SVD approximations to the matrix shown in
Figure~\ref{fig:zscores_vh0Sorted}.
For each of these approximations, I will compute the Frobenius norm of the
difference between the matrix and its truncated SVD approximation, and check
if this error is close to the sum of singular values in Eq.~\ref{eq:errorFNorm}.

Figure~\ref{fig:singularValues} plots the singular value of the matrix in
Figure~\ref{fig:zscores_vh0Sorted}. We see large reductions in singular values
upto the xth singular value. Therefore, we expect to see large improvements in
the approximation error as we increase the rank of the truncated SVD until rank
x, but not later.

\end{appendices}

% x weights
% x interpreation of vh0Sorted
%   . last response time at 31xx (add vline to heatmap)
%   . neurons sync to response times \in motor cortex layer ?? and striatrum  (STR, ,,,)
%   . neurons not sync to response times \in pallidum
% x svd interpretation as a sum of of rank 1 matrices
% . results with bin_size = 0.05

\end{document}
