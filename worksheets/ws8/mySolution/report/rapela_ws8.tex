\documentclass[12pt]{article}

\usepackage{natbib}
\usepackage{apalike}
\usepackage[hypertexnames=false,colorlinks=true,breaklinks]{hyperref}
\usepackage{graphicx}
\usepackage[shortlabels]{enumitem}
\usepackage{subcaption}
\usepackage{caption}
\usepackage{float}
\usepackage{amsmath}
\usepackage{amssymb}
\usepackage{amsthm}
\usepackage[title]{appendix}
\usepackage[margin=1in]{geometry}
\usepackage{verbatim}
\usepackage[many]{tcolorbox}

\newtheorem{definition}{Definition}
\newtheorem{lemma}{Lemma}
\newtheorem{theorem}{Theorem}

\tcolorboxenvironment{theorem}{
    colback=blue!5!white,
    boxrule=0pt,
    boxsep=1pt,
    left=2pt,right=2pt,top=2pt,bottom=2pt,
    oversize=2pt,
    sharp corners,
    before skip=\topsep,
    after skip=\topsep,
}

\tcolorboxenvironment{definition}{
    colback=blue!5!white,
    boxrule=0pt,
    boxsep=1pt,
    left=2pt,right=2pt,top=2pt,bottom=2pt,
    oversize=2pt,
    sharp corners,
    before skip=\topsep,
    after skip=\topsep,
}

\tcolorboxenvironment{lemma}{
    colback=blue!5!white,
    boxrule=0pt,
    boxsep=1pt,
    left=2pt,right=2pt,top=2pt,bottom=2pt,
    oversize=2pt,
    sharp corners,
    before skip=\topsep,
    after skip=\topsep,
}

\def\fig_width{3.5in}
\title{Report worksheet 8}
\author{Joaqu\'{i}n Rapela}

\begin{document}

\maketitle

\section*{Question 1: mutual information: one trial length}

Function \texttt{def H(x):} computes the entropy of the probability
distribution given in variable \texttt{x}. This distribution can be univariate,
e.g., \texttt{x=p(y)}, or bivariate, e.g., \texttt{x=p(y, z)}. Function
\texttt{def mi(C):} computes mutual information from the expression $I(X,
Y)=H(X)+H(Y)-H(X,Y)$, using the previous function \texttt{def H(x):} to
calculate the marginal and joint entropies.

Because $X$ and $Y$ are independent random variables, there mutual information
is zero. However, we don't get zero, since the code above uses the plugin
estimate of mutual information, which is biased above, and returns a positive
mutual information estimate.

\section*{Question 2: mutual information: multiple trial lengths}

Figure~\ref{fig:miEstimatesMultiTrial} shows the results of estimating mutual
information (using the plugin method) for different numbers of trials.

\begin{figure}[H]
    \begin{center}
        \href{https://www.gatsby.ucl.ac.uk/~rapela/neuroinformatics/2023/ws8/figures/mi_nRepeats_1_maxNTrials_100_p_0.500000_q_0.500000.html}{\includegraphics[width=5.5in]{../figures/mi_nRepeats_1_maxNTrials_100_p_0.500000_q_0.500000.png}}

        \caption{Estimates of mutual information, using the plugin method, for different number of trials.
        Generated with
        \href{https://github.com/joacorapela/neuroinformatics23/blob/master/worksheets/ws8/mySolution/code/scripts/doEstimateMI_multiTrials.py}{this}
        script using parameters \texttt{--n\_repeats 1}. Click on the image to see its
        interactive version.}

        \label{fig:miEstimatesMultiTrial}
    \end{center}
\end{figure}

\section*{Question 3: mutual information: multiple trial lengths}

As the number of trials increases, the plugin estimate of mutual information
should approach its true value. In agreement with this,
Figure~\ref{fig:miEstimatesMultiTrial} shows that, as the number of trials
increases, the estimated mutual information approches its true zero value.

For one trial the mutual information estimate is zero, because the probability
estimates for one trial are those of a deterministic random variable, and the
marginal and joint entropies of deterministic random variables are zero.

We cannot compute mutual information for an experiment with  zero trials.

\section*{Question 4: computing mutual information: multiple trial lengths, averaged}

Figure~\ref{fig:miEstimatesMultiTrialAveraged} shows averaged estimates of
mutual information across multiple trial lengths.

\begin{figure}[H]
    \begin{center}
        \href{https://www.gatsby.ucl.ac.uk/~rapela/neuroinformatics/2023/ws8/figures/mi_nRepeats_1000_maxNTrials_100_p_0.500000_q_0.500000.html}{\includegraphics[width=5.5in]{../figures/mi_nRepeats_1000_maxNTrials_100_p_0.500000_q_0.500000.png}}

        \caption{Averaged estimates of mutual information, using the plugin method, for different number of trials.
        Generated with
        \href{https://github.com/joacorapela/neuroinformatics23/blob/master/worksheets/ws8/mySolution/code/scripts/doEstimateMI_multiTrials.py}{this}
        script using parameters \texttt{--n\_repeats 1000}. Click on the image to see its
        interactive version.}

        \label{fig:miEstimatesMultiTrialAveraged}
    \end{center}
\end{figure}

\section*{Question 5: cross-validated mutual information: one trial length}

The code to estimate cross-validated mutual information is the same as the one
used above, with the exception that entropies are estimated with cross
validation. Two sets of probabilities are calculated, one set is used for the
probability component of the entropy, and the other set for the log-progability
component. Thus, if both estimates of probability were accurate, the
cross-valiated estimates should be identical to the non-cross-validated ones.

We get estimates smaller than zero because, differently from the plugin
estimation method, cross-validation estimation is biased below.

\section*{Question 6: cross-validated mutual information: multiple trial lengths}

Figure~\ref{fig:miEstimatesCVMultiTrial} shows cross-validated estimates of mutual
information for different numbers of trials.

\begin{figure}[H]
    \begin{center}
        \href{https://www.gatsby.ucl.ac.uk/~rapela/neuroinformatics/2023/ws8/figures/mi_cv_nRepeats_1_maxNTrials_100_p_0.500000_q_0.500000.html}{\includegraphics[width=5.5in]{../figures/mi_cv_nRepeats_1_maxNTrials_100_p_0.500000_q_0.500000.png}}

        \caption{Cross-Validated estimates of mutual information for different number of trials.
        Generated with
        \href{https://github.com/joacorapela/neuroinformatics23/blob/master/worksheets/ws8/mySolution/code/scripts/doEstimateMI_cv_multiTrials.py}{this}
        script using parameters \texttt{--n\_repeats 1}. Click on the image to see its
        interactive version.}

        \label{fig:miEstimatesCVMultiTrial}
    \end{center}
\end{figure}

\section*{Question 7: why do we get error in the function $\log_2$ during the estimation of the cross-validated mutual information?}

Because sometimes we compute $\log_2$ of zero values, since the
\texttt{nonzero} variable is set based on the probability used in the
probability part of the entropy (and not based on the probability used to
calculated the log-probability part of the entropy).

\section*{Question 8: computing mutual information: multiple trial lengths, averaged, cross-validated}

Figure~\ref{fig:miEstimatesCVMultiTrialAveraged} shows averaged cross-validated estimates of
mutual information across multiple trial lengths.

\begin{figure}[H]
    \begin{center}
        \href{https://www.gatsby.ucl.ac.uk/~rapela/neuroinformatics/2023/ws8/figures/mi_cv_nRepeats_1000_maxNTrials_100_p_0.500000_q_0.500000.html}{\includegraphics[width=5.5in]{../figures/mi_cv_nRepeats_1000_maxNTrials_100_p_0.500000_q_0.500000.png}}

        \caption{Averaged cross-validated estimates of mutual information for different number of trials.
        Generated with
        \href{https://github.com/joacorapela/neuroinformatics23/blob/master/worksheets/ws8/mySolution/code/scripts/doEstimateMI_cv_multiTrials.py}{this}
        script using parameters \texttt{--n\_repeats 1000}. Click on the image to see its
        interactive version.}

        \label{fig:miEstimatesCVMultiTrialAveraged}
    \end{center}
\end{figure}

\section*{Question 8: why the shape of the mutual information estimates in Fig.~\ref{fig:miEstimatesCVMultiTrialAveraged}}

Because cross-validated estimates of mutual information are biased below.

\end{document}
